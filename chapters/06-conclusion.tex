\chapter{Conclusion}


Inginious related:
\begin{itemize}
  \item Change container runtime to crun
  \item Could use btrfs for database course
  \item Things could get even better, but maybe not worth it
  \item Rootless and virtualization offers new possibilities of tasks
\end{itemize}


Things I learned:
\begin{itemize}
  \item First time dealing with such a big project
  \item First time I had to build my own learning resources from what I could find
  \item First time interacting with an open source community, felt really great to get their support
  \item It's hard to work on things your friends don't understand
  \item I knew it can be hard to present result when you don't have enough data, didn't know it could be hard when you have to much of it
\end{itemize}


Remaining questions:
\begin{itemize}
  \item There is a continious evolution of the different tools, it could be interesting to track, maybe create a benchmark tool more versatile that can be updated easily and run periodically?  (New realease of Kata Containers with updated Firecracker support from start of may hasn't been tested, Podman v2 is going to release soon)
  \item Check behaviour with real inginious containers
  \item Strange storage driver behavior for virtualization
  \item Check LXD virtualization solution
  \item Check LXD unpriviledged containers
  \item How good could be a viable custom-inginious container management solution
\end{itemize}
